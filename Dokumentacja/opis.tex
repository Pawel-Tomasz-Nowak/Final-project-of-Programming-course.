\documentclass{article}
\usepackage{polski}

\begin{document}
\begin{center}
\large Jendrek Natalia, Koczkodaj Natalia, Kowalczyk Dominik, Nowak Paweł, Smoła Szymon \\
\end{center}

\bigskip

\large Celem projektu naszej grupy jest stworzenie prostej gry mobilnej w koszykówkę. Zdecydowaliśmy się na ten temat ze względu na popularność tego typu gier oraz zafascynowania sportem członków drużyny.\\

\smallskip
Mechanika gotowej gry jest następująca: w pewnym miejscu planszy pojawia się armata, z której wystrzeliwana jest piłka do gry w koszykówkę. Miejsce będzie wybierane losowo, ale sensownie, tzn. piłka pojawić się może tylko w tych miejscach, z których wykonanie rzutu będzie możliwe. Celem gry jest trafienie piłki do kosza, co wraz ze zdobywaniem kolejnych punktów będzie coraz trudniejsze. Na początku gry kosz będzie nieruchomy, jednak gdy gracz będzie trafiał i przechodził na kolejne poziomy gry, zaczną pojawiać się utrudnienia, takie jak ruchomy kosz.\\

\smallskip
\large Pracę w trakcie realizowania projektu będziemy rozkładać równomiernie na wszystkich członków zespołu, delegując każdemu poszczególne zadania oraz konsultując progresy i potknięcia całą grupą. Do stworzenia gry potrzebować będziemy następujących bibliotek: Pygame, NumPy oraz Pathlib. \\


\section{Faza Druga}
\subsection{Aktualizacja planu na kolejne tygodnie pracy}

\smallskip
\large W najbliższych tygodniach planujemy najpierw zedytować klasę odpowiedzialną za tablicę oraz kosz, w celu umożliwienia aktualizacji współrzędnych tablicy (umożliwić jej poruszanie się). Następnie przystąpimy do zmian prędkości oraz tła, a także dodania licznika punktów i czasu u dołu ekranu. Na końcu rozważymy opcjonalne dodanie funkcji pauza, instrukcji na ekranie startowym oraz efektów dźwiękowych w różnych momentach rozgrywki.

\subsection{Zaktualizowany plan funkcjonalności gotowej aplikacji}

\smallskip
\large Po przeanalizowaniu naszego projektu zdecydowaliśmy się na unieruchomienie środka armaty, umożliwiając jedynie na zmianę nachylenia o maksymalnie 90 stopni w każdą stronę. Pozostawiamy ideę ruchomego kosza, który zacznie poruszać się wraz ze zwiększającą się ilością rzutów. Jednocześnie będziemy zmieniać również szerokość kosza (nie tablicy), utrudniając tym samym graczowi zdobycie potencjalnie bardzo dobrego wyniku. Planujemy, aby regulacji ulegała także prędkość piłki, którą wraz z biegiem gry będzie zwalniać. Pod armatą planujemy dodać również dynamiczny licznik punktów oraz czasu (opcjonalnie także ilość pozostałych rzutów do oddania). Wstępnie co określoną liczbę wystrzałów tło będzie ulegało zmianie. Po wykonaniu powyższych punktów w zależności od chęci oraz pozostałego czasu zostaną dodane efekty dźwiękowe i polepszony zostanie aspekt wizualny rozgrywki.  \\

\end{document}
