\documentclass{article}
\usepackage{polski}

\begin{document}
\begin{center}
\large Jendrek Natalia, Koczkodaj Natalia, Kowalczyk Dominik, Nowak Paweł, Smoła Szymon \\
\end{center}

\bigskip

\large Celem projektu naszej grupy jest stworzenie prostej gry mobilnej w koszykówkę. Zdecydowaliśmy się na ten temat ze względu na popularność tego typu gier oraz zafascynowania sportem członków drużyny.\\

\smallskip
Mechanika gotowej gry jest następująca: w pewnym miejscu planszy pojawia się armata, z której wystrzeliwana jest piłka do gry w koszykówkę. Miejsce będzie wybierane losowo, ale sensownie, tzn. piłka pojawić się może tylko w tych miejscach, z których wykonanie rzutu będzie możliwe. Celem gry jest trafienie piłki do kosza, co wraz ze zdobywaniem kolejnych punktów będzie coraz trudniejsze. Na początku gry kosz będzie nieruchomy, jednak gdy gracz będzie trafiał i przechodził na kolejne poziomy gry, zaczną pojawiać się utrudnienia, takie jak ruchomy kosz.\\

\smallskip
\large Pracę w trakcie realizowania projektu będziemy rozkładać równomiernie na wszystkich członków zespołu, delegując każdemu poszczególne zadania oraz konsultując progresy i potknięcia całą grupą. Do stworzenia gry potrzebować będziemy następujących bibliotek: Pygame, NumPy oraz Pathlib.

\end{document}
